\subsection{October}
\subsubsection{October 1st}
Today I learned the next lemma in the classification of finite abelian groups. To review, we currently are able to decompose finite abelian groups into direct products of their Sylow $p$-subgroups, so it remains to decompose $p$-groups into a direct product of cyclic $p$-groups.

The lemma is that if a finite abelian $p$-group $G$ has a unique smallest subgroup $H,$ then $G$ is actually cyclic. So as not to surprise the reader, we say that we're going to (strongly) induct on $|G|.$ Our base case of $|G|=p$ holds because Lagrange says that each non-identity element generates $G.$ So we may assume $|G|>p.$

We start by putting a name on $H.$ We note that by Cauchy that there is a (cyclic) subgroup of minimal order $p,$ so $H$ must have this order, and in fact $H$ must be made up of all the elements of order $p,$ for these each generate a subgroup of order $p.$ More succinctly, if we define $\varphi:g\mapsto g^p,$ then
\[\ker(\varphi)=\left\{g\in G:g^p=e\right\}.\]
Now, $\ker(\phi)$ consists exactly of the elements of order $p,$ all of which generate $H,$ so $\{e\}\subsetneq\ker(\phi)\subseteq H,$ so $H$ being smallest requires $H=\ker(\varphi).$ Note that $\ker(\varphi)\subsetneq G.$

The advantage of having homomorphism $\varphi$ is that the homomorphism theorem says that $\varphi(G)\cong G/\ker(\varphi),$ and $G/\ker(\varphi)$ is a smaller group than $G,$ so the inductive hypothesis lets us extract a coset $g\ker(\varphi)$ for which
\[\langle g\ker(\varphi)\rangle=G/\ker(\varphi)\]
generating $G/K.$ So we claim that $g$ also generates $G.$ We already know that every element $h\in G$ is in some coset $g^\bullet\ker(\varphi),$ so $h=g^\bullet k$ for $k\in\ker(\varphi).$ So we only need to prove that $\ker(\varphi)\subseteq\langle g\rangle.$ But $\langle g\rangle$ is nontrivial ($\langle g\ker(\varphi)\rangle$ is nontrivial), so it has $H=\ker(\varphi)$ is a subgroup, which is what we wanted.

\subsubsection{October 2nd}
Today I learned the definition of the semidirect product. For groups $H$ and $K,$ we assume that we have a homomorphism $K\to\op{Aut}(H)$ by $k\mapsto\varphi_k.$ (This mapping can be thought of as a way for $K$ to act on $H$ for a coarse multiplication.) Then $H\rtimes_\varphi K$ is defined on $H\times K$ so that multiplication is done by
\[(h_1,k_1)(h_2,k_2)=(h_1\varphi_{k_1}(h_2),k_1k_2).\]
Very quickly, in the trivial case where $\varphi_\bullet=\op{id},$ we see that $H\rtimes_\varphi K\cong H\times K,$ so we justify somewhat the name ``product.'' Showing that this group behaves is somewhat annoying, but it does. We get $(1,1)$ as our identity, but inverses are $(h,k)^{-1}=(\varphi^{-1}_k(h),k^{-1})$ to make the multiplication work. We intentionally avoid talking about associativity.

To further justify the name ``product,'' we see that the set of elements $(h,e)$ are isomorphic to $H$ because
\[(h_1,e)(h_2,e)=(h_1\varphi_e(h_2),ee)=(h_1h_2,e)\]
because we must have $\varphi_e=\op{id}.$ Similarly, we can embed $K$ into $H\rtimes_\varphi K$ by elements $k\mapsto(e,k).$

\subsubsection{October 3rd}
Today I learned the finish of the proof in the classification of finite abelian groups. We showed a few days ago that we can decompose any finite abelian group into a direct product of its Sylow $p$-subgroups, which left the case of $p$-groups. Then we proved the lemma that if a finite abelian $p$-group has a unique smallest subgroup, then it's cyclic.

We'll show that for $G$ a finite abelian $p$-group with cyclic subgroup $C$ of maximal order, we have that $G\cong C\times G'$ for some subgroup $G'\subseteq G,$ where the isomorphism is the canonical $(x,y)\mapsto xy.$ For closure, we show how to finish the proof assuming this. There is always a nontrivial cyclic subgroup for nontrivial $G,$ so applying this lemma repeatedly to $G$ and then $G'$ and then $G''$ and continuing down (until we're left with a trivial group) forms a progressively longer direct product
\[G\cong C\times C'\times C''\times\cdots\]
with progressively decreasing (in order) $G^{(\bullet)}$ remainder groups. But eventually the $G^{(\bullet)}$ will be trivial, completing the decomposition into a direct product of cyclic groups.

Once again, in order to not surprise the reader, we induct on $|G|.$ In the base case of $C=G,$ we can write $G\cong C\times\{e\},$ which is what we wanted. Note that this is forced in $|G|=p,$ which is cyclic generated by any nontrivial subgroup by Lagrange. So we may take $C\subsetneq G.$

As setup, $C\subsetneq G$ gives an element $h\in G\setminus C.$ Now, $h\ne e\in C$ lets us say that $\langle h\rangle$ has a subgroup of order $p$ (generated by $h^{\ord(h)/p}$), which we'll name $H.$ Our main player will be the canonical homomorphism
\[G\to G/H.\]
Unsurprisingly, we're going to use $\varphi$ to active our induction. Because $H$ is cyclic, we see that $C\cap H=\{e\},$ so $CH\cong C\times H,$ and $CH/H\cong(C\times H)/(\{e\}\times H)\cong C,$ so $CH/H$ is a pretty large cyclic subgroup of $G/H.$ How large? Well, the order of any element $gH$ in $G/H$ is no more than its order in $G$ is no more than $|C|,$ so $CH/H\cong C$ has maximal order in $G/H$ among cyclic subgroups.

So we trigger our induction and get to write $G/K\cong CH/H\times G_K'$ (with some hideous notation), for some subgroup $G_H'\subseteq G/K.$ Making this nicer to look at, we fix $G'$ to be the preimage of $G_H',$ and we naturally claim that $G\cong C\times G'.$ Note that $G'$ must contain the preimage of $e\in G_H',$ so $H\subseteq G',$ so we note that $G/H\cong CH/H\times G'/H$ implies
\[G=CH\cdot G'H=CG'.\]
But further, the fact we're working the canonical embedding $CH/H\times G'/H\to G/H$ lets us assert that $CH/H\cap G'/H=eH,$ so $CH\cap G'=\{e\}.$ So it follows from the above that $G=C\times G',$ which is what we wanted.

\subsubsection{October 4th}
Today I learned an interesting application of M\"obius inversion. The question extends the classic light bulb problem: in an infinite sequence of numbered light bulbs ($1$-indexed), person $n$ flips the switches corresponding to all light bulbs divisible by $n$ for $n\in\ZZ^+.$ We want to know which light bulbs are on after everyone is done. Well, these are exactly the ones for which
\[\sum_{d\mid n}1=d(n)=\prod_{p\mid n}(\nu_p(n)-1)\]
is odd. But this requires each individual $\nu_p(n)-1$ to be odd, or each $\nu_p(n)$ to be even, so light bulb $n$ is on if and only if $n$ is square.

There is some interesting structure when we stop using every single person. Of course, given a set of people, we can determine which light bulbs are on at the end by computing the parity of $\#\{d\mid n:d\text{ is used}\}.$ Adding in some notation, we can assign bits to each person in $\{p_k\}$ so that $1$ means the person is used and $0$ means not. Then the sum we want the parity of
\[\#\{d\mid n:p_d=1\}=\#\{p_d:d\mid n\}=\sum_{d\mid n}p_d=(1*p)(n),\]
which is a bit nicer. A slightly more interesting question is that, given a pattern of light bulbs to be on, can we reverse it to get a sequence of people turning those light bulbs on?

This is not terribly hard; naturally, we can do it by an induction. Suppose we have a sequence of bits $\{b_k\}$ that we want our light bulbs to match, and we want to assign bits $\{p_k\}$ to our sequence of people. We claim that the first $n$ people-bits make the first $n$ light bulbs match up with $\{b_k\},$ in fact uniquely, which provides an affirmative answer to the question. We do this by induction. When $n=0,$ there is nothing to prove.

Then assume we can do this uniquely with $n=N$ bulbs; we show $n=N+1.$ We start off the sequence $\{p_k\}_{k=1}^N$ generating the first $N$ bulbs. Then we claim $p_{N+1}$ is determined uniquely. Indeed, compare the parity of $b_{N+1}$ and
\[\sum_{\substack{d\mid N+1\\d<N+1}}p_d.\]
If they match, then $b_{N+1}$ must be off. This will not affect any previous bits $b_k,$ so we indeed cover all bulbs up to $N+1.$ Else if the parities don't match, then $b_{N+1}$ must be odd to correct. Again, we cover all bulbs to $N+1$ and are forced in our choice. This finishes.

What's weird is that the above inductive process gives me the feeling of a finite Fourier transform. In particular, it feels as if we should be able to read off $p_k$ with some summation not requiring the above inductive argument. Relabel our sequences $b(n)$ and $p(n)$ so that we want
\[b(n)\equiv\sum_{d\mid n}p(d)\pmod2.\]
I guess this is equivalent to
\[(-1)^{b(n)}=\prod_{d\mid n}(-1)^{p(d)}.\]
Then M\"obius inversion tells us this is equivalent (!) to
\[(-1)^{p(n)}=\prod_{d\mid n}(-1)^{\mu(d)b(d)},\]
which is what we wanted. As an aside, I'm used to proving the product form of M\"obius inversion by logarithms, but it should still work in the above by just choosing a branch, probably. In any case, we can always just do the old proof of plugging one equation into the other to show the implications.

\subsubsection{October 5th}
Today I learned about some characters of $(\QQ,+),$ as always from Keith Conrad. Namely, these are group homomorphisms $\QQ\to S^1,$ where $S^1$ is the unit circle in $\CC.$ For brevity, we write $\widehat\QQ=\op{Hom}(\QQ,S^1).$ We can generate ``a lot'' of characters by considering various embeddings of $\QQ$ into its local fields. So, for example, we have that, for any $s\in\RR,$
\[q\mapsto e^{(2\pi is)q}\]
is a fine character by exponent rules. The correct way to look at these homomorphisms is as the restriction of our homomorphisms $\RR\to S^1$ to $\QQ.$ We include the $2\pi$ factor so that we can remark this is ``really'' a homomorphism from $\QQ/\ZZ\to S^1$ which has been extended to all of $\QQ.$ In the same way, we can say the correct way to look at this is as a restriction of the obvious homomorphism $\RR/\ZZ\to S^1.$

We can extend this idea to local files other than $\RR.$ Namely, fix $p$ a finite prime, and we'll look at $\QQ_p.$ And as motivated by $\RR,$ we'll really try to look at $\QQ_p/\ZZ,$ but this is a bit unnatural, so we actually want to look at $\QQ_p/\ZZ_p.$ Explicitly, we have a surjection $\QQ_p\to\QQ_p/\ZZ_p$ which ``extracts'' the negative-power terms:
\[\sum_{n=-N}^\infty a_np^n\in\QQ_p\longmapsto\sum_{n=-N}^{-1}a_np^n\in\QQ_p/\ZZ_p.\]
As in $\RR,$ we will denote this $p$-adic fractional part by $\{\bullet\}_p.$ The nice thing about $\{\bullet\}_p,$ though, is that we have a natural embedding $\QQ_p/\ZZ_p\to\QQ$ because these elements are finite sums of rationals. So we're allowed to claim that, for some $s\in\QQ_p,$
\[q\mapsto e^{2\pi\{sq\}_p}\]
is a homomorphism $\QQ\to S^1.$ We'll show here that $q\mapsto e^{2\pi\{q\}_p}$ is a homomorphism; afterwards, we'd just need to replace $q$ with $sq$ to finish.

Essentially, to make this homomorphism work, we need to know that $\{x+y\}_p=\{x\}_p+\{y\}_p$ in $\QQ/\ZZ$ in the same way that $\{x+y\}=\{x\}+\{y\}$ in $\RR/\ZZ.$ But note
\[\{x+y\}_p-\{x\}_p-\{y\}_p=-(x+y-\{x+y\}_p)+(x-\{x\}_p)+(y-\{y\}_p)\in\ZZ_p,\]
so the equality surely holds in $\QQ_p/\ZZ_p.$ However, $\{x+y\}_p-\{x\}_p-\{y\}_p$ is then a $p$-adic integer created as a finite sum of elements of $\QQ_p$ with finite base-$p$ expansion; it follows we have a $p$-adic integer with finite base-$p$ expansion, so $\{x+y\}_p-\{x\}_p-\{y\}_p\in\ZZ$ after all.

As an aside, it's fun to note that this homomorphism, say $\psi_p,$ is essentially the one that fits this commutative diagram.
\begin{center}
    \begin{tikzcd}
    \mathbb Q_p/\mathbb Z_p \arrow[dd, "\{\bullet\}_p"'] \arrow[rrd, "\psi_p", dashed] &  &     \\
                                                                             &  & S^1 \\
    \mathbb Q/\mathbb Z \arrow[rru, "\exp"']                                 &  &    
    \end{tikzcd}
\end{center}
However, writing $\QQ/\ZZ=\bigoplus_p\QQ_p/\ZZ_p$ (with the natural $\{\bullet\}_p$ inclusions), tells us that there is a unique homomorphism $\QQ/\ZZ\to S^1$ which commutes with any of our choices $\psi_p$ (including the multiplicative scaling). I think the categorical perspective is a bit satisfying.

\subsubsection{October 6th}
Today I learned a loose connection between Fourier analysis and theory of Dirichlet series and convolutions. The correct context to abstract both of these ideas is in abstract Fourier analysis. Namely, take $G$ a group; I think the correct hypotheses are locally compact abelian topological group, but I don't care that much. Write $\widehat G=\op{Hom}(G,\CC^\times)$ for its characters. Then we say the Fourier transform of a function $f:G\to G$ is another function $\mathcal F(f):\widehat G\to G$ by
\[\mathcal F(f)(\chi)=\int_Gf(x)\overline{\chi(x)}\,d\mu(x)\]
for some measure $\mu$ on our $G.$

This formula looks scary, but it's quite familiar. In the context of Fourier analysis over $\RR,$ our characters $\RR\to\CC^\times$ are $\exp$ and friends (you can make other choice-like characters, but preserving the topological properties of $\CC$ forces). However, we can quite easily enumerate these characters by $x\mapsto e^{2\pi isx}$ for some scale factor $s\in\CC.$ So we can actually express $\mathcal F(f)$ by
\[\mathcal F(f)(s)=\int_\RR f(x)e^{-2\pi isx}\,dx.\]
And viola, the Fourier transform appears. Abstract theory beind Fourier analysis (which I don't understand) can give us typical properties we like; e.g., applying the Fourier transform twice gives back the original function. While we're here, we remark that we have a Fourier function convolution by $\mathcal F(f*g)=\mathcal F(f)\mathcal F(g).$ This will be helpful momentarily.

The nice observation, now, is to apply this to $(\NN,\cdot)$ (it's not a group, but it still roughly works) in order to obtain theory of Dirichlet series and convolutions. What are our characters $\NN\to\CC^\times$? We're going to cheat a bit and require these to also be characters over $(\RR^+,\cdot),$ presumably so that continuity and other things work out. (I don't entirely understand why this is the correct thing to do.) In particular, we can write, for character $\chi,$
\[\chi(n)=\chi(e)^{\log n}=n^{\log\chi(e)}.\]
In particular, we can parameterize all of our characters by $s\in\CC$: it's the one taking $n\mapsto\chi(n)=n^s.$ Then our Fourier transform looks like
\[\mathcal F(f)(s)=\int_\NN f(n)n^{-s}\,dn.\]
An integral over $n$ is really an infinite sum. Namely, we see
\[\mathcal F(f)(s)=\sum_{n=1}^\infty\frac{f(n)}{n^s}.\]
And viola, the Dirichlet series for $f$ appears. Of note is that $\mathcal F(\op{id})=\zeta$ and $\mathcal F(\delta)=1,$ where $\delta$ is the $1$-indicator.

Relating this to convolutions, we still write $f*g$ as our function satisfying $\mathcal F(f*g)=\mathcal F(f)\cdot\mathcal F(g).$ For example, this tells us that $\delta$ is the identity of this operation. And more generally, taking products,
\[\sum_{n=1}^\infty\frac{(f*g)(n)}{n^s}=\mathcal F(f*g)=\mathcal F(f)\mathcal F(g)=\left(\sum_{n=1}^\infty\frac{f(n)}{n^s}\right)\left(\sum_{n=1}^\infty\frac{g(n)}{n^s}\right).\]
In particular, expanding tells us that
\[\sum_{n=1}^\infty\frac{(f*g)(n)}{n^s}=\sum_{n=1}^\infty\left(\sum_{ab=n}f(a)g(b)\right)\frac1{n^s}.\]
So this motivates defining $(f*g)(n)=\sum_{ab=n}f(a)g(b),$ the definition of the Dirichlet convolution! And then once we see $\mu*\op{id}=\delta,$ we get M\"obius inversion, as usual.

\subsubsection{October 7th}
Today I learned about adeles $\AA_\QQ$ in order to construct the characters over $(\QQ,+).$ The definition, exactly, is
\[\AA_\QQ=\{(a_\infty,a_2,a_3,a_5,\ldots)\}\subseteq\RR\times\prod_p\QQ_p\]
such that all but finitely many of the $a_p\in\QQ_p$ are in $\ZZ_p.$ (The product is over finite primes.) The condition for all but finitely many of the $a_p\in\QQ_p$ in $\ZZ_p$ is not quite direct product (which allows free reign) and not quite coproduct (which forces all but finitely to be $0$). The reasoning for this restricted product, as far as I can tell, is to make the embedding
\[q\in\QQ\longmapsto(q,q,\ldots)\in\AA_\QQ\]
actually behave. Because $r\in\QQ,$ we can't force all of the $a_p$ to be in $\ZZ_p,$ but surely we can't have infinitely many primes in the denominator of $r,$ so only finitely of the $a_p$ should be permitted to live in $\QQ_p\setminus\ZZ_p.$

What $\AA_\QQ$ is ``really'' doing here is that it supercharges our local-global intuition. It's often helpful to solve a problem over our local fields $\RR$ or $\QQ_p$ (say, a Diophantine or something else where analysis is nice) and then see how that information can be translated up to $\QQ.$ But it might not be clear which local fields exactly we want to use, or we might have to use some specific subset. Well, $\AA_\QQ$ lets us use all of these at the same time.

In context of the characters we had two days ago, we note that we can write something like
\[\frac{\AA_\QQ}{\RR\times\prod_p\ZZ_p}\cong\{0\}\times\bigoplus_p\QQ_p/\ZZ_p\cong\QQ/\ZZ,\]
where that's a $\bigoplus$ because $a_p\in\ZZ_p$ gets thrown to $0.$ We generated our characters as over $\QQ_p\ZZ_p,$ so we can see that, at the very least, $\AA_\QQ$ will do a nice job of smooshing characters in each individual $\QQ_p/\ZZ_p$ into characters over $\QQ.$ In reality, we define the characters over $\QQ$ pretty much as we would expect, writing  $a\in\AA_\QQ$ gives
\[\Psi_a(q)=\psi_\infty(a_\infty a)\prod_p\psi_p(a_pq),\]
where $\psi_\infty(q)=e^{-2\pi iq}$ and $\psi_p(q)=e^{2\pi i\{q\}_p}$ from last time. To show that $\Psi_\bullet$ makes sense, we note that the condition that all but finitely many $a_p$ are in $\ZZ_p$ implies that all but finitely of the $qa_p$ are in $\ZZ_p.$ But this means $\psi_p(a_pq)=1$ for all but finitely many $p,$ so $\Psi_\bullet$ is actually a finite product, so $\Psi_\bullet$ does make sense.

Some nice facts about $\Psi_\bullet$ include that it's actually a homomorphism $\AA_\QQ\to\widehat\QQ,$ which can be seen by expanding $\Psi_a\Psi_b=\Psi_{a+b}.$ Further, for a rational adele $(r,r,\ldots)\in\QQ^\infty\cap\AA_\QQ,$ we have $\Psi_r=\op{id}.$ Indeed, after plugging in the definition of the various $\psi_\bullet,$ we need to show
\[r-\sum_p\{r\}_p\in\ZZ.\]
To show this, we show that it is in $\ZZ_p$ for every $p,$ forcing the denominator to have no primes. Indeed, we can read this sum as $r-r_p-(\text{stuff with denominators not divisible by }p),$ which is indeed in $\ZZ_p.$

\subsubsection{October 8th}
Today I learned a nice argument that if all roots of a polynomial $P(x)\in\ZZ[x]$ have magnitude no more than $1,$ then they are all either $0$ or roots of unity. Factor $P(x)$ over $\CC$ as
\[P(x)=\prod_{k=1}^n(x-\alpha_k).\]
Note that all of the $\alpha_\bullet$ are algebraic integers. The main pllayer in this proof is the sequence of polynomials
\[P_m(x)=\prod_{k=1}^n\left(x-\alpha_k^m\right).\]
Essentially, it is necessary and sufficient to show that, eventually, these polynomials cycle their roots---if they're all roots of unity or $0,$ then taking the least common multiple of all the periods will give us an $M$ so that $P_{M+1}(x)=P(x),$ for example.

For this, we show there are only finitely many options for the $P_m(x),$ which we can do by showing that the coefficients live in $\ZZ[x]$ and bounded. For $\ZZ[x],$ note that $\ZZ=\overline\ZZ\cap\QQ,$ so it suffices to show that $P_m$'s coefficients live in both $\overline\ZZ$ and $\QQ.$ The former isn't difficult: observe that a full expansion of
\[P_m(x)=\prod_{k=1}^n\left(x-\alpha_k^m\right)\]
will give coefficients that are all finite sums of finite products of the $\alpha_\bullet^m.$ But $\overline\ZZ$ is a ring, and we know each $\alpha_\bullet\in\overline\ZZ,$ so these coefficients must be in $\overline\ZZ$ as well.

To show that the coefficients are in $\QQ,$ we think about $P_m(x)$ in the splitting field $K/\QQ$ of $P(x).$ This extension is Galois, so we can show that for each $\sigma\in\op{Gal}(K/\QQ)$ fixes the coefficients of $P_m(x).$ Because this is the splitting field of $P(x),$ $\sigma$ must permute the roots $\{\alpha_\bullet\}$ of $P(x)$ (as a multiset), so $\sigma$ also permutes the roots $\{\alpha_\bullet^m\}$ of $P_m(x).$ In particular, $\sigma(P_m(x))=P_m(x),$ which is what we wanted.

So far we know that $P_m(x)\in\ZZ[x].$ To show bounded coefficients, we again look at
\[P_m(x)=\prod_{k=1}^n\left(x-\alpha_k^m\right)\]
and imagine a full expansion. For the coefficient of $x^d,$ there are $\binom nd$ ways to choose $x$s, and this gives $\binom nd$ terms summing to the coefficient of $x^d.$ But each term is a product of $\alpha_\bullet^m$s, which all have magnitude no more than $1,$ so each of the $\binom nd$ has magnitude no more than $1.$ So we get to bound each coefficient, using the triangle inequality by
\[\binom nd\cdot|1|=\binom nd.\]
Importantly, this bound is independent of $m.$

Now we get the payoff for the above lemmas. The coefficients of each $P_m(x)$ are all in $\ZZ[x]$ and bounded in magnitude, so there are only finitely many polynomials that the $P_m(x)$ can be. In particular, we must eventually have $a$ and $b$ give $P_a(x)=P_{a+b}(x).$ Waving my hands around the multiplicity problem a bit, we see that the multiset of roots of $P_a(x)$ and $P_{a+b}$ must be the same, there is a permutation $\sigma\in S_n$ such that
\[\alpha_\bullet^{a+b}=\alpha_{\sigma(\bullet)}^a\]
for any of the $\alpha_\bullet.$ In particular, we have that
\[\alpha_\bullet^{a+n!b}=\alpha_{\sigma^{n!}(\bullet)}^a=\alpha_\bullet^a\]
because the order of $\sigma$ divides $|S_n|=n!.$ But the only roots of $x^{a+n!b}-x^a$ are $0$ and roots of unity, so each $\alpha_\bullet$ must be $0$ or a root of unity.

\subsubsection{October 9th}
Today I learned that every character $\chi:\QQ\to S^1$ can indeed by expressed as $\Psi_a$ for some adele $a\in\AA_\QQ.$ This is very classic local-global principle: we're going to decompose $\chi$ into a product of character-like things $\chi_\infty$ and $\chi_p$ for all of our places, and then we'll finish by showing our character-like things are actually characters. This will let us identify $\chi$ with some adele, finishing.

For each $\psi_p,$ we have that $\psi_p(1a_p)=e^{2\pi i\{a_p\}_p}$ is a root of unity, so any non-root-of-unity contribution to $\chi$ must come from $\chi_\infty.$ In particular, let $\chi(1)=e^{-2\pi ia_\infty}$ so that $\chi_\infty(q)=e^{-2\pi ia_\infty q}$ satisfies $\chi(1)=\chi_\infty(1).$ Then for any $q=\frac mn,$ we have that
\[\frac\chi{\chi_\infty}(q)^n=\frac\chi{\chi_\infty}(nq)=\frac\chi{\chi_\infty}(m)=\frac\chi{\chi_\infty}(1)^m=1,\]
so indeed, $\frac\chi{\chi_\infty}$ has image only equal to roots of unity, dictated by the denominator. With that in mind, we define $\frac\chi{\chi_\infty}(q)=e^{2\pi i\theta}$ for $\theta\in\QQ$ so that $\chi_p(q)=e^{2\pi i\{\theta\}_p}.$ In particular, $-\theta+\sum_p\{\theta\}_p\in\ZZ$ implies that
\[\frac\chi{\chi_\infty}(q)=e^{2\pi i\theta q}=\prod_pe^{2\pi i\{q\}_p}=\prod_p\chi_p(q).\]
Namely,
\[\chi(q)=\chi_\infty(q)\prod_p\chi_p(q).\]
So we have decomposed $\chi$ into character-like $\chi_\bullet$s. This ends the global step.

We defined $\chi_\infty(q)=e^{2\pi ia_\infty q}=\psi_\infty(a_\infty q),$ so we'll finish by showing that $\chi_p(q)=\psi_p(a_pq)$ for some $a_p\in\ZZ_p.$ This will tell us that each character $\chi$ is associated with some $\Psi_a$ for $a\in[0,1)\times\prod_p\ZZ_p\subsetneq\AA_\QQ,$ completing the proof of the surjection of $\Psi_\bullet.$ We begin by studying $\chi_p(1)$ in order to construct $a_p.$ For any $\bullet\in\NN,$ recall that
\[\chi_p\left(1/p^\bullet\right)^{p^\bullet}=\chi_p(1)=1,\]
so it follows $\chi_p\left(1/p^\bullet\right)=e^{2\pi ic_\bullet/p^\bullet}$ for some $c_\bullet\in\ZZ/p^\bullet\ZZ.$ We'd like to have $\{c_\bullet/p^\bullet\}_p=\{1a_p/p^\bullet\}_p,$ so we had better have $a_p\equiv c_\bullet\pmod{p^\bullet}.$ Because we're constructing an element of $\ZZ_p,$ this modular information is enough to construct $a_p$ as a Cauchy sequence, but we need to show that this is really a Cauchy sequence. Well, $\chi_p\left(1/p^{\bullet+1}\right)^p=\chi_p\left(1/p^\bullet\right),$ so $c_{\bullet+1}\equiv c_\bullet\pmod{p^\bullet}.$ Thus,
\[a_p=\{c_1,c_2,\ldots\}\in\ZZ_p\]
is indeed a Cauchy sequence in $\ZZ_p,$ which we'll say converges to $a_p.$ Now we have to extend this $a_p$ to work for all $q=\frac mn.$ To control the denominator, fix $n=p^\bullet n'$ with $(n',p)=1.$ This gives
\[\chi_p(q)^{n'}=\chi_p\left(m/np^\bullet\right)^{n'}=\chi_p\left(1/p^\bullet\right)^m=e^{2\pi ia_pm/p^\bullet}.\]
We would like this exponent to have a $\{a_pq\}_p$ in it, and indeed
\[\frac{a_pm}{p^\bullet}\equiv m\left\{a_p/p^\bullet\right\}_p=\left\{a_pm/p^\bullet\right\}_p=\{a_pqn'\}_p\equiv n'\{a_pq\}_p\pmod\ZZ\]
Here we have used $\{x\}_p+\{y\}_p=\{x+y\}_p$ repeatedly. So we see that $\chi_p(q)^{n'}=\left(e^{2\pi i\{a_pq\}_p}\right)^{n'}.$ But $\chi_p(q)$ is a $p$th root of unity, so the quotient is a $p$th root of unity as well as an $n'$th root of unity, forcing the quotient to be $1$ because $(n',p)=1.$ It follows $\chi_p(q)=e^{2\pi i\{a_pq\}_p},$ which is what we wanted.

\subsubsection{October 10th}
Today I learned the finale to the classification of $\widehat\QQ.$ Namely, that $\widehat\QQ\cong\AA_\QQ/\QQ.$ We begin by remarking that the identification between $\chi\in\widehat\QQ$ with adeles in $[0,1)\times\prod_p\ZZ_p$ was in fact unique. Indeed, $a_\infty$ was forced by $\chi(0),$ and each $a_p$ had its $\ZZ/p^\bullet\ZZ$ information uniquely read by each $\chi(1/p^\bullet).$ And in reverse, each $a_\infty\in[0,1)$ uniquely talks about $\chi(0),$ and the $a_p$ in this set cover all possible $\chi(1/p^\bullet)$ possibilities, information which extends to all of $\QQ$ roughly using our arguments from before.

It more or less remains to show, then, that every adele $a\in\AA_\QQ$ is a rational adele away from an element of $[0,1)\times\prod_p\ZZ_p.$ The main trick is to start by dealing with the $p$-adic problems by looking at
\[-\sum_p\{a_p\}_p.\]
Because all but finitely many of the $a_p$ belong to $\ZZ_p,$ this is a finite sum. It follows that adding the rational (adele associated with this rational) to $a$ will give an adele which is always in $\ZZ_p.$ To force us into $[0,1),$ then, we consider
\[-\floor{a_\infty-\sum_p\{a_p\}_p}-\sum_p\{a_p\}_p.\]
Note that the new integer added here does not change the status of $a_p\in\ZZ_p,$ and as for $a_\infty,$ the floor is chosen so that we are left with $\{\bullet\}\in[0,1)$ for a mess $\bullet.$ This shows that, more or less, we may write
\[\AA_\QQ/\QQ\cong\QQ+[0,1)\times\prod_p\ZZ_p.\]
We formalize this slightly.

To finish, we recall that our mapping $\Psi_\bullet:\AA_\QQ\to\widehat\QQ$ was a surjective (shown yesterday) homomorphism and has $\QQ$ in the kernel; we close by showing the kernel is $\QQ.$ Fix some adele in the kernel $a.$ Writing $a=q+a'$ with $q$ a rational adele and $a'\in[0,1)\times\prod_p\ZZ_p$ given by the above, we see
\[\Psi_a=\Psi_q\Psi_{a'}=\Psi_{a'}.\]
However, the uniqueness of our identification of characters $\widehat\QQ$ with $[0,1)\times\prod_p\ZZ_p$ implies that for $a$ (and thus $a'$) to be in the kernel, we must have $a'=0.$ In particular, $a=q$ was a rational adele all along, so $\QQ$ is precisely the kernel.

It follows that
\[\widehat\QQ\cong\AA_\QQ/\QQ,\]
which is what we wanted.

\subsubsection{October 11th}
Today I learned that the commutator subgroup (subgroup generated by $aba^{-1}b^{-1}$) of $S_n$ is $A_n,$ for $n\ge3.$ Let the commutator subgroup be $C=\langle \sigma\tau\sigma^{-1}\tau^{-1}:\sigma,\tau\in S_n\rangle,$ and we'll show $C=A_n.$

In one direction, $C\subseteq A_n$ is not hard. Simply note that for any $\sigma\tau\sigma^{-1}\tau^{-1}\in C,$ we have
\[\op{sgn}\left(\sigma\tau\sigma^{-1}\tau^{-1}\right)=\op{sgn}(\sigma)\op{sgn}(\tau)\op{sgn}{\sigma}^{-1}\op{sgn}(\tau)^{-1}=1\]
because $\op{sgn}$ is a homomorphism. From this we see $\sigma\tau\sigma^{-1}\tau^{-1}\in A_n,$ so $C\subseteq A_n.$ It remains to show $A_n\subseteq C.$

As a lemma, we show that $A_n$ is generated by $3$-cycles; for clarity, this is why $n\ge3$ matters. Indeed, $A_n$ is comprised of all even permutations, so it is generated by permutations made up of two transpositions, say $(a,b)(c,d).$ We do casework on the overlap.
\begin{enumerate}[label=(\roman*)]
    \item If $(a,b)=(c,d),$ then this is the identity, which is generated by $3$-cycles.
    \item If only one letter in each is the same, say $b=c$ with $a\ne d,$ then we see
    \[(a,b)(b,d)=(a,b,d)\]
    is already a $3$-cycle.
    \item If all letters are different, then we write
    \[(a,b)(c,d)=(a,b)\circ(b,c)(b,c)\circ(c,d)=[(a,b)(b,c)]\circ[(b,c)(c,d)],\]
    which reduces to the previous case.
\end{enumerate}
It follows that the generators of $A_n$ can be expressed as products of $3$-cycles, so $A_n$ can be expressed as products of $3$-cycles.

With the lemma in mind, it suffices to express $3$-cycles as a commutator, which will force $A_n\subseteq C$ because we just showed $A_n$ is generated by those $3$-cycles. And indeed,
\[(a,b,c)=(a,c)(b,c)(a,c)(b,c)=(a,c)(b,c)(a,c)^{-1}(b,c)^{-1}\in C.\]
This completes the proof that $A_n\subseteq C,$ and so we are done.

\subsubsection{October 12th}
Today I learned that the leading digits of Fibonacci numbers $F_n$ follow Benford's Law. (Yesterday I learned $9^k$ follows Benford's Law.) The main lemma is that $\{\log F_n\}$ is equidistributed in $[0,1],$ where we are taking $\log$s base $10.$ (The base isn't important, so I prefer not to write it.) In particular, we use Binet's Formula to write
\[\log F_n=\log\left(\frac{\varphi^n-(-1)^n\varphi^{-n}}{\sqrt5}\right)=n\log\varphi+\log\left(1-(-1)^n\varphi^{-2n}\right)-\log\sqrt5.\]
Note $\log\sqrt5$ is a translation and will not affect the distribution$\pmod1,$ so we may ignore it. We're left with
\[n\log\varphi+\log\left(1-(-1)^n\varphi^{-2n}\right).\]
We already know that $\{n\log\varphi\}$ is equidistributed by the Equidistribution theorem, so we hope that the other term is sufficiently small; we show that it is. I guess we have to show that $\log\varphi$ is irrational, but $\log\varphi=\frac pq\in\QQ$ would imply that $\varphi^q=10^p.$ However, no $\varphi^\bullet$ is an integer by adding in their conjugate power, which is less than $1,$ so this is a contradiction.

Note that as $n\to\infty,$ we have $\log\left(1-(-1)^n\varphi^{-2n}\right)\to0,$ so we may fix a bound $N'$ so that $n>N'$ forces $\log\left(1-(-1)^n\varphi^{-2n}\right)<\varepsilon,$ for an arbitrary $\varepsilon.$ It follows that
\begin{align*}
    & \left\{n<N:\{n\log\varphi+\log\left(1-(-1)^n\varphi^{-2n}\right)\}\in(x,y)\right\} \\
    \subseteq & \left\{N'<n<N:\{n\log\varphi\}\in(x+\varepsilon,y-\varepsilon)\right\},
\end{align*}
and
\begin{align*}
    & \left\{n<N:\{n\log\varphi+\log\left(1-(-1)^n\varphi^{-2n}\right)\}\in(x,y)\right\} \\
    \supseteq & \left\{N'<n<N:\{n\log\varphi\}\in(x-\varepsilon,y+\varepsilon)\right\},
\end{align*}
where we are somewhat ignoring looping errors with $x$ and $y$ close to $1$---we could just send $\varepsilon$ sufficiently small so these don't occur. It follows that
\[\lim_{N\to\infty}\frac{\#\left\{n<N:\{n\log\varphi+\log\left(1-(-1)^n\varphi^{-2n}\right)\}\in(x,y)\right\}}N\in[y-x-2\varepsilon,y-x+2\varepsilon]\]
because we already know $\{n\log\varphi\}$ is equidistributed. Taking $\varepsilon\to0$ implies that the limit is $y-x,$ implying that $\log F_n$ is equidistributed as well.

To finish, we show that the equidistributed nature of $\log F_n$ implies that the leading digit of $F_n$ follows Beford's Law. Indeed, the leading digit of $F_n$ is $d\in[0,10)\cap\ZZ$ if and only if
\[d\cdot10^\bullet\le F_n\le(d+1)\cdot10^\bullet\]
for some suitable power of $10$ named $10^\bullet.$ Taking logs, this is equivalent to
\[\bullet+\log d\le\log F_n\le\bullet+\log(d+1).\]
However, we know that $\bullet$ is an integer, so taking fractional parts is legal, implying that this is equivalent to
\[\log d\le\{\log F_n\}\le\log(d+1).\]
In particular, we require $\log F_n\in[\log d,\log(d+1)],$ which occurs with probability $\log(d+1)-\log d$ because we just showed that $\log F_n$ is equidistributed. But this is Benford's Law, so we are done here.

\subsubsection{October 13th}
Today I learned that the connection between $\left(\frac\bullet p\right)$ and quadratic forms doesn't need quadratic reciprocity. For a nontrivial example, we use $K=\QQ(\sqrt{-7}).$ Let $\theta=\frac{1+\sqrt{-7}}2$ so that $\mathcal O_K=\ZZ[\theta].$ Note that our norm is
\[\op N(a+b\theta)=\left|\left(a+\frac b2\right)+\left(\frac b2\right)\sqrt{-7}\right|^2=a^2+ab+2b^2.\]
As an aside, $\theta$ has minimal polynomial $x^2-x+2,$ which looks similar.

We need to know that $\ZZ[\theta]$ is Euclidean, with that norm. For $a,b\in\ZZ[\theta]$ with $b\ne0,$ we need $q$ and $r$ with $a=bq+r$ with $\op N(r)<\op N(b).$ As usual, we can just write
\[\frac ab=q+\frac rb,\]
where we want $\op N(r/b)<1.$ and then let $q$ be $\frac ab$ rounded to the nearest element of $\ZZ[\theta].$ Then the components of $\frac rb$ are each less than or equal to $\frac12,$ so
\[\op N(r/b)\le\left(\frac12\right)^2+\left(\frac12\right)\left(\frac12\right)+2\left(\frac12\right)^2=1,\]
with equality if and only if the components of $r/b$ are both $\pm\frac12$ and the same sign. But being the midpoint of this parallelogram and rounding down to the far corner is not optimal. Simply, $\op N\left(\frac12+\frac12\theta\right)$ can be turned into $\op N\left(-\frac12+\frac12\theta\right)<1$ by incrementing $q$ by $1,$ and we're outside of the equality case and safe. So $\ZZ[\theta]$ is Euclidean.

Euclidean matters because it tells us that $\ZZ[\theta]$ is a principal ideal domain. So $(p)$ splits in $\ZZ[\theta]$ if and only if $(p)=(\alpha)(\beta),$ which holds if and only if $p=\alpha\beta$ for some $\alpha,\beta\in\ZZ[\theta]$ not units. Taking norms, we have that $\op N(\alpha)\in\{1,p,p^2\}$ (the norm is positive), but $\op N(\alpha)=1$ implies $\alpha$ is a unit, and $\op N(\alpha)=p^2$ implies $\op N(\beta)=1$ so that $\beta$ is a unit. But then $\op N(\alpha)=p,$ so
\[p=a^2+ab+2b^2\]
for some $a+b\theta=\alpha\in\ZZ[\theta].$

On the other hand, we can examine prime-splitting without thinking about principal ideal domains. Note $p$ is prime in $\ZZ[\theta]$ if and only if $(p)$ is a prime ideal, or $(p)$ is a maximal ideal because $\ZZ[\theta]$ is Dedekind. So $p$ remains prime if and only if $\ZZ[\theta]/(p)$ is a field. Abusing notation and glossing over some details, we note that
\[\frac{\ZZ[\theta]}{(p)}\cong\frac{\ZZ[x]/\left(x^2-x+2\right)}{(p)}\cong\frac{\ZZ[x]/(p)}{\left(x^2-x+2\right)}\cong\frac{\FF_p[x]}{\left(x^2-x+2\right)}.\]
So $\ZZ[\theta]/(p)$ is a field if and only if $\FF_p[x]/\left(x^2-x+2\right)$ is a field as well, which we know to be true if and only if $x^2-x+2$ is irreducible$\pmod p$ from theory of finite fields. But this is a quadratic, so it's irreducible if and only if it has a root, and because its roots are $\frac{1\pm\sqrt{-7}}2,$ it has roots if and only if $p=2$ or $\left(\frac{-7}p\right)\ne-1.$ Because $\left(\frac{-7}2\right)=1,$ we may use $\left(\frac{-7}p\right)\ne1$ as our condition.

Combining these two results, we see $p$ factors nontrivially in $\ZZ[\theta]$ if and only if
\[p=a^2+ab+2b^2\quad\iff\quad\left(\frac{-7}p\right)\ne-1.\]
And we have not used any quadratic reciprocity to show this; we only need to know that $\mathcal O_{\QQ(\sqrt d)}$ is a principal ideal domain for this machinery to work. Quadratic reciprocity would tell us $p=a^2+ab+2b^2$ for some $a,b\in\ZZ$ if and only if $\left(\frac p7\right)\ne-1$ if and only if $p=7$ or $p\equiv1,2,4\pmod7.$

\subsubsection{October 14th}
Today I learned that for a number field $K,$ there exists an extension $L/K$ such that every ideal $I\subseteq\mathcal O_K$ has $I\mathcal O_L$ principal (!), from \href{https://math.stackexchange.com/questions/294150/motivation-behind-the-definition-of-ideal-class-group?rq=1}{this post}. This follows, roughly, from the finiteness of the class group.

We begin with a lemma, showing that principal ideals do indeed form a class. Namely, we already know that principals are closed under multiplication, but we can show that $(\alpha)\cdot I=(\beta)$ implies $I=(\gamma)$ for some $\gamma,$ where we are dealing with ideals in $\mathcal O_K.$ For this, note that the equality gives us an element $\gamma\in I$ such that
\[\alpha\gamma=\beta.\]
Indeed, there is some $\alpha\gamma_\alpha\in(\alpha)$ and $\gamma_I\in I$ such that $\alpha\gamma_\alpha\gamma_I=(\beta),$ but surely $\gamma=\gamma_\alpha\gamma_I\in I.$ Now we know $(\gamma)\subseteq I.$ For the reverse, fix any $i\in I$ and write
\[\alpha i=\beta j=\alpha\gamma j.\]
Because we live inside an integral domain, we may cancel so that $i=\gamma j\in(\gamma).$ So $I$ is indeed principal, and this is a class. (This discussion is roughly isomorphic to saying $I=(\alpha)/(\beta)$ in the fractional ideals.)

Now we proceed in steps, more or less. We began by showing that there exists an extension $L$ for which $I\mathcal O_L$ for a particular ideal $I.$ For this, let $n=|\op{Cl}_K|$ so that $I^n$ must belong to the class of principal ideals ($[I]$ has order dividing $n$). But then $I^n=(\gamma),$ and we claim $I\mathcal O_L=(\sqrt[n]{\gamma})$ in $L=K(\sqrt[n]{\gamma}),$ which will prove this step. Indeed, simply note that
\[(I\mathcal O_L)^n=I^n\mathcal O_L=(\gamma)=(\sqrt[n]\gamma)^n.\]
This lets us use unique prime factorization for a quick finish: for any prime $\mf p\subseteq\mathcal O_L,$ we have that $n\nu_\mf p(I\mathcal O_L)=n\nu_\mf p((\sqrt[n]\gamma)).$ Thus, $\nu_\mf p(I\mathcal O_L)=\nu_\mf p((\sqrt[n]\gamma)),$ so the two ideals have the same prime factorization and therefore must be equal.

To finish, we note that the class group is finite, so we consider $L=K(\sqrt[n]{\gamma_1},\ldots,\sqrt[n]{\gamma_n}),$ where we adjoin a $\sqrt[n]{\gamma_\bullet}$ for a representative of each ideal class. Now we claim all ideals are principal in $L.$ Indeed, fix some $I\subseteq\mathcal O_L$ an ideal, and let $J$ be its representative from the class group so that $J\mathcal O_L=(\sqrt[n]{\gamma_\bullet}).$ We know there exist $\alpha$ and $\beta$ such that
\[\alpha I=\beta J\]
as an alternate definition of the class group. But then
\[(\alpha)\cdot I\mathcal O_L=(\beta)\cdot J\mathcal O_L=(\beta\sqrt[n]{\gamma_\bullet}).\]
It remains to show that $I\mathcal O_L$ is principal from this, which is an application of the lemma we began with.

What I like about this result is that it kind of speaks to some of this history Professor Kedlaya gave when talking about ideals. Namely, when we're trying to fix unique prime factorization with respect to bad things like
\[2\cdot3=6=(1+\sqrt{-5})\cdot(1-\sqrt{-5}),\]
Kummer's idea was to create ``ideal numbers'' $\mf p_1,\mf p_2,\mf p_3,\mf p_4$ so that this factorization dissolved into
\[(\mf p_1\mf p_2)(\mf p_3\mf p_4)=(\mf p_1\mf p_3)(\mf p_2\mf p_4).\]
Professor Kedlaya mentioned in passing that we technically always can extend into a field where all of our ideals are indeed principal, and here we kind of see that. Namely, we would like to conjure $\mf p_1$ from a far-off field as the legitimate greatest common divisor of $2$ and $1+\sqrt{-5},$ the generator of $(2,1+\sqrt{-5}),$ and this generator does exist in a far-off field. The problem, of course, is that the far-off field need not be a principal ideal domain, so we run into somewhat of a recursive problem---we'd like to extend again to make ideals principal, and so on.

\subsubsection{October 15th}
Today I learned the proof for the existence of non-Euclidean principal ideal domains. The easiest example that could possibly work is $\QQ(\sqrt{-19})\cap\overline\ZZ,$ which is $\ZZ[\theta]$ for $\theta=\frac{1+\sqrt{-19}}2.$ It's a classical application of the Minkowski bound to show that $\ZZ[\theta]$ has trivial class group and therefore is a principal ideal domain. We show here that it is not a Euclidean domain.

The approach is to use the fact that all Euclidean domains satisfy the ``universal side divisor criterion,'' and then show that $\ZZ[\theta]$ does not. Namely, fix $R$ a Euclidean domain with norm $d:R\to\NN.$ Then for each $a,b\in R$ with $b\ne0,$ there exist $q$ and $r$ with $d(r)<d(b)$ satisfying
\[a=bq+r.\]
For the universal side divisor criterion, we say that we can give some $x$ not a nonzero non-unit which minimizes $d$ among other nonzero non-units. But then for any $a\in R,$ our division says
\[a=xq+r\]
forces $d(r)<d(x),$ so $r\in R^\times\cup\{0\}.$ In particular, $R/xR$ has representatives which are all units or zero, which is exactly the universal side divisor criterion. As an aside, whenever there is a Euclidean norm $d,$ it is always possible to define a multiplicative norm $d',$ so $x$ could be defined as an element achieving the smallest norm above $1.$ We don't need to say this in order to get $R/xR$ having representatives all units or zero, as above.

Now we apply this to $\ZZ[\theta].$ In particular, we claim that no such $x$ exists, but before we do so, we take a little time to classify units. For this, note that $\op N(a+b\theta)$ evaluates to
\[\op N\left(\left(a+\frac b2\right)+\left(\frac b2\right)\sqrt{-19}\right)=\left(a+\frac b2\right)^2+\frac{19b^2}2.\]
An element is a unit if and only if the norm is $1.$ Now, if $b\ge1,$ then the norm is at least $\frac{19}2>1,$ so we must have $b=0.$ Then the norm is $a^2,$ for which $a=\pm 1$ are both necessary and sufficient. In particular, $\ZZ[\theta]^\times=\{\pm1\}.$

To finish the proof, we claim that there is no $x$ for which $\ZZ[\theta]/x\ZZ[\theta]$ only has representatives from $\ZZ[\theta]^\times\cup\{0\}=\{-1,0,1\}.$ Before using this property of our $x\in\ZZ[\theta],$ note that the ring $\ZZ[\theta]/x\ZZ[\theta]$ sends $\theta$ to some coset, so
\[\theta^2-\theta+5=0\]
implies that $\ZZ[\theta]/x\ZZ[\theta]$ ought have a root of the polynomial $p(\theta)=\theta^2-\theta+5.$

Now we assume for the sake of contradiction that $x$ is satisfies the universal side divisor criterion. In particular, $\ZZ[\theta]/x\ZZ[\theta]$ is fully represented by $\{-1,0,1\}.$ It surely has more than one element (else $x\ZZ[\theta]=\ZZ[\theta],$ forcing $x$ to be a unit), so it either has $2$ or $3$ elements; i.e., it is isomorphic to either $\FF_2$ or $\FF_3,$ telling us the structure of our ring for free. But $p(\theta)=\theta^2-\theta+5$ has no roots in both$\pmod2$ and$\pmod3,$ for it evaluates to $p(0)=5,$ $p(1)=5,$ and $p(2)=7.$ It follows $\theta$ can't have a coset in $\ZZ[\theta]/x\ZZ[\theta],$ which is our contradiction.

What I find a bit amusing about this proof is that exactly what makes $\ZZ[\theta]$ a principal ideal domain is what makes it not a Euclidean domain: that the minimal polynomial 
\[p(\theta)=\theta^2-\theta+5\]
produces quite a few primes in its image. On one hand, this makes the Dedekind-Kummer factorization algorithm for ideals quickly conclude all of our small primes are inert, giving trivial class group. But on the other hand, it forces $\ZZ[\theta]/x\ZZ[\theta]$ for any $x\in\ZZ[\theta]$ to have quite a few elements in order to accommodate for a root of $p(\theta).$

As an aside, this proof doesn't carry over to, say, $\QQ(\sqrt{-11})\cap\overline\ZZ$ because our minimal polynomial is $x^2-x+3,$ which reduces$\pmod3.$ However, it does carry over to the other Heegner numbers at least $19,$ which is nice. In the other direction, this utterly fails in real quadratic fields because they have infinite unit groups, so our size arguments about $R/xR$ fail spectacularly.

I guess as foreshadowed, we could say this is a relationship between our unit group $\ZZ[\theta]^\times$ and the structure of unique prime factorization. Namely, on a high level, we could say that the universal side divisor criterion was really just a tool to let us talk about Euclidean domains in terms of their unit group. It happens that we understand the unit group of $\ZZ[\theta]$ super well, which is what allowed us to more closely study Euclidean division.

\subsubsection{October 16th}
Today I learned the proof for the lower bound that comparison sorts must take at least $\Omega(n\log n)$ time. This is not terribly hard, so I guess today's entry will be short. In short, a comparison sort is very roughly a query of one bit of information among the $n!$ total possible permutations. So purely in terms of information theory, we need at least $\log_2n!=\Theta(n\log n)$ queries, which is what we wanted.

Purely for the sake of more words, we add some of the details. Essentially, imagine there is an oracle with the array $(a_\bullet)$ and the sorting algorithm queries the oracle with indices $(i,j)$ for comparison information $a_i>a_j.$ Afterwards, the algorithm should output the (indexed, perhaps) sorted order of the array. We'll say querying the oracle has constant-time cost, as in real life. The claim is that at least $\Omega(n\log n)$ queries to the oracle are necessary to sort.

To see this, replace the oracle with an evil genie, as usual. The evil genie will decide what the array is while the algorithm queries. Begin with the set of elements $\{1,\ldots,n\}$; the genie will keep track of the full pool of $n!$ permutations. When the algorithm queries the genie $(i,j),$ the genie checks all currently possible permutations of $\{1,\ldots,n\}$ (taking into account previous comparison information given to the algorithm) and chooses the comparison information of $(i,j)$ permitting the most permutations of $\{1,\ldots,n\}$. Namely, there are two possibilities:
\begin{itemize}
    \item $a_i>a_j,$ or
    \item $a_i\le a_j.$
\end{itemize}
One of these will have a number of associated permutations at least the other, and the genie will make that choice, dividing the set of possible remaining permutations of $\{1,\ldots,n\}$ into no less than half.

If the algorithm stops and asserts a sorted array (of indices, say) before there is one permutation remaining in the genie's pool, then the genie can just choose one of the permutations unequal to the one provided by the algorithm, and the algorithm will have sorted incorrectly on that permutation.

It follows that the number of queries to the evil genie must at least succeed in dividing the pool of $n!$ permutations down to a single permutation. But each query divides the pool by no more than half, so after $q$ queries, we will have divided the pool into no more than
\[\frac{n!}{2^q}\]
permutations. For this to be no more than $1,$ we must have taken $2^q\ge n!,$ so $q\ge\log_2n!$ queries. Adjusting for constants, this means we need $\Omega(\log n!)$ queries.

It remains to show $\Omega(\log n!)=\Omega(n\log n).$ The slick way to do this is to ignore constant terms and say
\[\log n!=\sum_{k=1}^n\log k\ge\sum_{k>n/2}^n\log\frac n2\ge\left(\frac n2-1\right)\log\frac n2=\frac12n\log n-n\log2-\frac12\log n+\frac12\log2.\]
Dividing by $n\log n$ and taking $n\to\infty$ gives a ratio of
\[\frac12-\frac{\log2}{\log n}-\frac1{2n}+\frac{\log2}{n\log n}\longrightarrow\frac12.\]
It follows $\Omega(\log n!)=\Omega(n\log n),$ which completes the proof.

Because I know some analytic number theory, I remark that Abel summation implies
\[\log n!=\sum_{k=1}^n\log k\cdot1=n\log n-\int_1^n\frac{\floor x}x\,dx.\]
For the integral, we can write it as
\[\int_1^n\frac xx\,dx-\int_1^n\frac{\{x\}}x\,dx=n+O\left(\int_1^n\frac1x\,dx\right)=n+O(\log n).\]
It follows $\log n!=n\log n+n+O(\log n).$

\subsubsection{October 17th}
Today I learned a little bit about Eisenstein's criterion, in hopes of finding a polynomial for which Eisenstein's criterion fails for any transformation. Namely, Keith Conrad shows \href{https://kconrad.math.uconn.edu/blurbs/gradnumthy/totram.pdf}{here} that if the minimal polynomial for $\alpha$ satisfies Eisenstein's criterion with $p$ (``is $p$-Eisenstein''), then $p$ totally ramifies in $\QQ(\alpha).$ The converse is also true---if $p$ totally ramifies in $K,$ then there exists an $\alpha$ with $K=\QQ(\alpha)$ such that the minimal polynomial for $\alpha$ is $p$-Eisenstein---but we don't show that here.

Indeed, fix a root $\alpha$ of a monic polynomial $f\in\ZZ[x]$ of degree $n$ which is $p$-Eisenstein and therefore irreducible. (If the polynomial wasn't monic, we could multiply it by the leading coefficient enough times until it was without changing the $p$-Eisenstein condition or the generated field $\QQ(\alpha).$) Explicitly, we know that with
\[f(x)=\sum_{k=0}^na_kx^k,\]
we have $a_n=1,$ each $a_k$ is divisible by $p$ for $k\in[0,n),$ and $p^2\nmid a_0.$

We're going to show that $p$ is totally ramified in $K=\QQ(\alpha).$ We suspect there is only one prime above $p,$ so let $\mf p$ be a prime above $p.$ We want to show $(p)=\mf p^n,$ but for now let $(p)=\mf p^eI$ for some $\mathcal O_K$-ideal $I.$ The hard part of this argument is showing $e\ge n.$

Now we return to talking about $f(x).$ The main idea is to reduce
\[\alpha^n+\sum_{k=1}^na_k\alpha^k=f(\alpha)\equiv0\pmod{\mf p^\bullet}\]
for some large power $\mf p^\bullet$; it turns out most of these terms are already divisible by a large power. Being $p$-Eisenstein tells us that $a_k\in(p)$ for $k\in[0,n),$ which gives $a_k\in\mf p^e$ for free. So, reducing gives that
\[\alpha^n\equiv0\pmod{\mf p^e},\]
and in particular $\alpha\in\mf p.$ But then we can push this to $a_k\alpha^k\in\mf p^{e+1}$ for $k\in[1,n),$ so we know
\[\alpha^n+a_0\equiv0\pmod{\mf p^{e+1}}.\]

Quickly, the other $p$-Eisenstein condition is that $p^2\nmid a_0,$ which will tell us that $a_0\not\in\mf p^{e+1}.$ Indeed, this is equivalent to $\mf p^{e+1}\nmid(a_0),$ so it's enough to show that $\nu_\mf p(a_0\mathcal O_K)=e.$ We may let $a_0=pa$ where $p\nmid a,$ and then expanding this as a unique prime factorization tells us
\[a_0\mathcal O_K=p\mathcal O_K\cdot a\mathcal O_K=\mf p^eI\cdot a\mathcal O_K.\]
The above has successfully factored $\mf p^e$ out of $a_0\mathcal O_K,$ so we want to know that the remaining factors have no $\mf p.$ Well, $I$ is coprime by definition, and $\mf p\mid a\mathcal O_K$ is equivalent to $a\in\mf p$ implies $a\in\mf p\cap\ZZ=p\ZZ,$ contradicting the definition of $a.$

Returning to the argument, we see $\alpha^n\equiv-a_0\pmod{\mf p^{e+1}},$ so $a_0\not\in\mf p^{e+1}$ implies that $\alpha^n\not\in\mf p^{e+1}.$ This means $\mf p^{e+1}\nmid(\alpha)^n,$ but $(\alpha)$ is divisible by at least one $\mf p\ni\alpha,$ so we get $\mf p^{e+1}\nmid\mf p^n.$ It follows $e+1>n,$ or $e\ge n.$ This finishes the hard part of the argument.

To finish, we use the fundamental identity. Indeed, note
\[n=\sum_{\mf q/p}e(\mf q/p)f(\mf q/p)\ge nf(\mf p/p)+\sum_{\substack{\mf q/p\\\mf q\ne \mf p}}e(\mf q/p)f(\mf q/p)\ge n\]
forces $f(\mf p/p)=1$ and removes the possibility of any other primes above $p.$ Thus, $p\mathcal O_K=\mf p^n,$ which is total ramification.

\subsubsection{October 18th}
Today I learned how to generate an irreducible polynomial which can't be proven with the standard tricks behind Eisenstein's Criterion. We begin by creating a field extension $K/\QQ$ such that $K$ has no totally ramified primes.

We claim that $K=\QQ(\sqrt2,\sqrt5)$ does the trick. Indeed, fix a rational prime $p\in\ZZ,$ and we'll show that $p$ is not totally ramified. We show this by considering the subfields $\QQ(\sqrt2)$ and $\QQ(\sqrt5).$ On one hand, note that $\QQ(\sqrt2)\cap\overline\ZZ=\ZZ[\sqrt2],$ and
\[\disc\left(\ZZ[\sqrt2]\right)=\det\begin{bmatrix}
    1 & \phantom-\sqrt2 \\
    1 & -\sqrt2 \\
\end{bmatrix}^2=(2\sqrt2)^2=8.\]
It follows that if $p\ne2,$ then $p$ is not ramified in $\QQ(\sqrt2).$ In particular, for any $\mf P$ of $\QQ(\sqrt2,\sqrt5)$ over $\mf p$ of $\QQ(\sqrt2)$ over $p$ of $\ZZ,$ we have that
\[e(\mf P/p)=e(\mf P/\mf p)e(\mf p/p)\le\left[\QQ(\sqrt2,\sqrt5):\QQ(\sqrt2)\right]\cdot1=2<4,\]
so $p$ is not totally ramified in $\QQ(\sqrt2,\sqrt5).$ Now, on the other hand, note that $\QQ(\sqrt5)\cap\overline\ZZ=\ZZ\left[\frac{1+\sqrt5}2\right],$ and
\[\disc\left(\ZZ\left[\textstyle\frac{1+\sqrt5}2\right]\right)=\det\begin{bmatrix}
    1 & \frac{1+\sqrt5}2 \\
    1 & \frac{1-\sqrt5}2 \\
\end{bmatrix}^2=(\sqrt5)^2=5.\]
It follows that if $p\ne5,$ then $p$ is not ramified in $\QQ(\sqrt5).$ Using the same logic as before, we fix any prime $\mf P$ of $\QQ(\sqrt2,\sqrt5)$ over $\mf p$ of $\QQ(\sqrt5)$ over $p$ of $\ZZ$ so that
\[e(\mf P/p)=e(\mf P/\mf p)e(\mf p/p)\le\left[\QQ(\sqrt2,\sqrt5):\QQ(\sqrt5)\right]\cdot1=2<4,\]
so $p$ is not totally ramified in $\QQ(\sqrt2,\sqrt5)$ again. It follows that no prime other than $2$ can be totally ramified in $\QQ(\sqrt2,\sqrt5)$ from $\QQ(\sqrt2),$ and no prime other than $5$ can be totally ramified in $\QQ(\sqrt2,\sqrt5)$ from $\QQ(\sqrt5),$ so no prime at all can be totally ramified in $\QQ(\sqrt2,\sqrt5).$

To finish, we claim that $P(x)=\boxed{x^4-14x^2+9}=\left(x^2-7\right)^2-40$ is not $p$-Eisenstein for any prime $p,$ no matter how we transform it like $P(ax+b).$ Indeed, suppose for the sake of contradiction that $P(ax+b)$ is indeed $p$-Eisenstein, for some integers $a$ and $b$ with $a\ne0.$ Note that $P(x)$ is the minimal polynomial for $\sqrt2+\sqrt5$ (it's a quartic with $\sqrt2+\sqrt5$ as a root), so it follows $P(ax+b)$ is the minimal polynomial for
\[\alpha=a\sqrt2+a\sqrt5+b.\]
However, by our discussion yesterday, it follows that $p$ is totally ramified in $\QQ(\alpha)=\QQ(\sqrt2+\sqrt5).$ But we just showed that $\QQ(\sqrt2+\sqrt5)=\QQ(\sqrt2,\sqrt5)$ has no totally ramified primes! So we have our contradiction, and we're done here.

I guess I might want to show that $\QQ(\alpha)=\QQ(\sqrt2+\sqrt5)$ and $\QQ(\sqrt2+\sqrt5)=\QQ(\sqrt2,\sqrt5).$ Well, $\sqrt2+\sqrt5=\frac1a(\alpha-b)$ shows the first equality, and the second one holds because
\[\sqrt2=\frac{(\sqrt2+\sqrt5)^3-11\sqrt2}6,\]
and we can get $\sqrt5$ by writing $\sqrt2+\sqrt5-\sqrt2.$ The finer details are really not terribly interesting.

As an aside, I think my $P(x)$ is (accidentally) also reducible$\pmod p$ for every rational prime $p.$ Indeed, we can check that it's $(x+1)^4\pmod2,$ and then for all other primes $p,$ I think (not super sure)
\[\left[\mathcal O_K:\ZZ[\sqrt2+\sqrt5]\right]=2,\]
so the Dedekind-Kummer factorization algorithm for ideals says that the splitting of $p$ can be determined by factoring $P(x)\pmod p.$ But the Galois group of $K=\QQ(\sqrt2,\sqrt5)$ is $\ZZ/2\ZZ\times\ZZ/2\ZZ,$ which isn't cyclic, so no prime $p$ may remain inert in $\mathcal O_K.$ (An inert prime forces $e(\mf p/p)=r=1,$ so the decomposition field is $\QQ$ and the inertial field is $K,$ so we must have $G\cong D\cong D/E$ is cyclic.)  It follows that $P(x)\pmod p$ can never remain irreducible$\pmod p.$

\subsubsection{October 19th}
Today I learned about \href{https://en.wikipedia.org/wiki/Midy's_theorem}{Midy's Theorem}. The statement is more magical than the proof. Basically, if the fraction $\frac ap\in\QQ\setminus\ZZ$ (for prime denominator $p$) has an even period when expanded in base $b,$ then the sum of the first and second halves of the period is of the form $b^k-1.$ By way of example, $\frac1{13}=0.\overline{076923},$ and so
\[076+923=\boxed{999}.\]
Similarly, $\frac{10}{137}=0.\overline{07299270},$
\[0729+9270=\boxed{9999}.\]

Anyways, as promised, the proof is not terribly exciting, but it simultaneously touches a lot of low-machinery elementary number theory, which is cute. There is a clean generalization of this to when we can divide the decimal expansion into any number of equal parts (not just $2$), but we won't bother with this. Suppose that we have $\frac ap$ satisfying the conditions so that
\[\frac ap=(0.\overline{a_1\cdots a_{2n}})_b=\frac A{b^{2n}-1},\]
where $A=\overline{a_1\cdots a_{2n}}_b.$ The clever part of this proof, really, is combining the divisibility condition on the period with the actual fact that it is the period. Namely, to use the divisibility, write
\[\frac ap=\frac A{\left(b^n+1\right)\left(b^n-1\right)}.\]
But because the period of $\frac ap$ is $2n,$ we have that $p$ cannot divide into $b^n-1,$ for then the period would no more than $k.$ (In particular, the base-$b$ expansion of $\frac 1p$ would be $\frac{b^n-1}p.$) But $p$ must divide the denominator of the right-hand side somehow, so it must divide into $b^n+1,$ implying that
\[\frac A{b^n-1}=a\cdot\frac{b^n+1}p\in\ZZ.\]
This means that $A\equiv0\pmod{b^k-1}.$

To finish, we then combine the above with the divisibility by $9$ trick. We write
\[A=\overline{a_1\cdots a_{2n}}=\underbrace{(\overline{a_1\cdots a_n})_b}_{A_1}\cdot b^n+\underbrace{(\overline{a_{n+1}\cdots a_{2n}})_b}_{A_2},\]
where $A_1$ and $A_2$ are the first and second halves of the base-$b$ expansion. Now, it remains to show that $A_1+A_2=b^n-1.$ The above shows that $b^n-1$ divides $A_1+A_2\equiv A,$ so we just need bounding information.

On one hand, they can't both be $0,$ for then the period would be $1,$ so $A_1+A_2>0.$ On the other hand, the fact that each has $k$ digits implies that $A_1$ and $A_2$ are each less than $b^k-1,$ so $A_1+A_2<2\left(b^n-1\right).$ The equality follows.

\subsubsection{October 20th}
Today I learned a proof for a slightly weaker version of Cohn's irreducibility criterion from \href{https://mast.queensu.ca/~murty/murty.pdf}{Murty}. Quite surprisingly, there is pretty much no number theory involved; it's pretty much entirely bounding. Fix our polynomial $f(x)=a_nx^n+\cdots+a_0\in\ZZ[x].$ The bounding constant of interest is
\[H=\max_k|a_k/a_n|.\]
We're not going to state the exact statement until the end of the proof.

We begin by showing that all roots $\alpha$ of $f$ satisfy $|\alpha|\le H+1.$ Note that $|\alpha|<1$ makes this immediate. Otherwise, $f(\alpha)=0$ implies
\[\alpha^n=-\frac1{a_n}\left(a_{n-1}\alpha^{n-1}+\cdots+a_0\right),\]
so
\[|\alpha|^n=\left|\sum_{k=0}^{n-1}\frac{a_k}{a_n}\alpha^k\right|\le\sum_{k=0}^{n-1}H|\alpha|^k=H\cdot\frac{|\alpha|^n-1}{|\alpha|-1}<H\cdot\frac{|\alpha|^n}{|\alpha|-1}.\]
Rearranging gives the desired $|\alpha|<H+1.$

To finish, assume (not for the sake of contradiction) $f(x)$ is irreducible, factored by $f(x)=g(x)h(x)$ with $g(x),h(x)\in\ZZ[x].$ We're going to introduce the exact criterion's condition later. We see that we may write $g(x)=\gamma\prod_k(x-\alpha_k)$ with $\gamma\ge1.$ Note roots $\alpha_k$ are also roots of $f(x),$ so for $x\ge H+2,$ we may bound
\[|g(x)|=|\gamma|\prod_{k=1}^{\deg g}|x-\alpha_k|\ge\prod_{k=1}^{\deg g}(|x|-|\alpha_k|)>1.\]
The same holds for $h(x).$

We structured the proof this way so far in order to emphasize the lack of number theory we've done so far---everything has been bounding. Only now do we bring in any number theory. If we fix an integer $n\ge H+2,$ we note that $f(n)=g(n)h(n)$ makes a nontrivial factorization for $f(n)$ because $g(n),h(n)>1$ from the above work. It follows that $f(n)$ must be composite. So we can say that $f(x)$ is irreducible if there exists an $n\ge H+2$ such that $f(n)$ is prime, and that's all.

It is conjectured that irreducible polynomials produce ``a lot'' of primes (say, see the Bateman-Horn conjecture), even for these large (actually pretty small) values of $x,$ so what I like about this irreducibility criterion is that it actually does a pretty good job of deciding irreducibility. Namely, compute $H+2$ and then try a whole bunch of consecutive values of $x$ while testing for primes. For example, the polynomial $x^2-14x^2+9$ from two days ago has $H+2=16,$ and $20^4-14\cdot20^2+9=154409,$ which is prime. So irreducibility follows.

Notably, this is still not sharp because of polynomials like $x^2+x+2$ being irreducible but always even. But I'm still content that Cohn's criterion works for most sane polynomials. According to the Bunyakovsky conjecture, the condition we need to add is that $\gcd(f(n))=1$ over all $n,$ which is quickly checked by computing the greatest common divisor of the first few values.

\subsubsection{October 21st}
Today I learned the correct context for the statement last week that if $P(x)\in\ZZ[x]$ only has roots which have magnitude less than or equal to $1,$ then all of those roots are roots of unity. This is actually a reformulation of the statement the multiplicative-to-additive mapping $\log:K^\times\to\RR^{r+s}$ (taken over $\mathcal O_K^\times$) has kernel the roots of unity, which is quite nice. Indeed, suppose $\alpha\in\mathcal O_K^\times$ is in the kernel. Then, applying the mapping, we have that
\[\log|\sigma(\alpha)|=0\]
for each embedding $\alpha.$ Namely, all of the Galois conjugates of $\alpha$ each have $|\sigma(\alpha)|=1.$ It follows that the minimal (monic) polynomial of $\alpha,$ which we name $P(x)\in\ZZ[x],$ has all of its roots (the Galois conjugates of $\alpha$) with magnitude equal to $1.$

It follows that the kernel of the $\log$ mapping can be identified with the roots of monic polynomials $P(x)\in\ZZ[x]$ all of whose roots have magnitude $1.$ But we've already studied these and found that they are contained in the set of roots of unity, so we're done here.

I guess I haven't shown the reverse implication that all roots of unity are sent to the kernel, but this isn't hard. Embeddings must take roots of unity (with finite multiplicative order) to other roots of unity (in order to preserve order), so all embeddings will stay on the unit circle and therefore have magnitude $1.$ So all roots of unity certainly are in the kernel.

\subsubsection{October 22nd}
Today I learned that the trace pairing over finite fields gives all possible $\FF_q$-linear maps $\FF_{q^r}\to\FF_q,$ from \href{https://sites.math.rutgers.edu/~sk1233/courses/finitefields-F13/intro.pdf}{here}. We begin by noting that the the Galois group
\[\op{Gal}(\FF_{q^r}/\FF_q)=\langle\sigma_q:\alpha\mapsto\alpha^q\rangle.\]
It follows that the trace looks like
\[\op{Tr}(\alpha)=\sum_{\sigma\in\op{Gal}(\FF_{q^r}/\FF_q)}\sigma(\alpha)=\sum_{k=0}^{r-1}\alpha^{q^k}.\]
This doesn't simpliyfy further in any nice way. As usual, we can check that $\op{Tr}:\FF_{q^r}\to\FF_q$ by noting that the Latin square property of groups implies that the above sum is held constant by any automorphism $\sigma.$ We could also say more explicitly that $\FF_q$ is the set of elements satisfying $x^q=x,$ which we can check manually because
\[\op{Tr}(\alpha)^q=\left(\sum_{k=0}^{r-1}\alpha^{q^k}\right)^q=\sum_{k=0}^{r-1}\alpha^{q^{k+1}}=\op{Tr}(\alpha),\]
where the last equality holds because $\alpha^{q^r}=\alpha.$

Again as usual, $\op{Tr}$ provides an $\FF_q$-linear map from $\FF_{q^r}\to\FF_q.$ Both of these can be seen from the definition using automorphisms or by manually expanding, so we won't show that here. What is more interesting is the proof that the image is all of $\FF_q.$ Indeed, the dimension of the image is at most $1,$ so we just have to show that the image is at least dimension $1.$ But $\op{Tr}(\alpha)=0$ if and only if $\alpha$ is a root of
\[\op{Tr}(x)=\sum_{k=0}^{r-1}x^{q^k},\]
a polynomial of degree $q^{r-1}<q^r.$ It follows (by Lagrangs) that $\op{Tr}$ has at most $q^{r-1}$ roots, so it is nonzero somewhere, so its image has at least dimension $1.$ This is what we wanted.

Because we have a well-behaved linear map, we can use it to define the inner product
\[\langle x,y\rangle\longmapsto\op{Tr}(xy),\]
which we can quickly check to be bilinear. This induces quite a few linear maps $\langle\gamma,\bullet\rangle,$ or $x\mapsto\op{Tr}(\gamma x).$ How many? We claim that all linear maps $\FF_{q^r}\to\FF_q$ have this form. For an upper bound, note that tracking where basis elements go tells us that there are $q^r$ total linear maps. It remains to show that the trace pairing induces this many maps.

To be explicit, our pairing induces mappings
\[\gamma\longmapsto(x\mapsto\op{Tr}(\gamma x)).\]
Now, this works for any $\gamma\in\FF_{q^r},$ so there are $q^r$ total maps induced here. So it is enough to show that these are unique; i.e., that this is injective. Indeed, if $\gamma$ and $\gamma'$ induce the same mappings so that we want to show $\gamma=\gamma'.$ Taking the difference implies that
\[x\mapsto\op{Tr}((\gamma-\gamma')x)=0\]
is identically $0.$ If $\gamma-\gamma'$ is nonzero, then we can set $x=(\gamma-\gamma')^{-1}\alpha$ where $\op{Tr}(\alpha)\ne0,$ which exists by the above discussion. So because $\op{Tr}((\gamma-\gamma')x)$ really is identically $0,$ we must have $\gamma-\gamma'=0,$ implying that $\gamma=\gamma'.$ This is what we wanted.

\subsubsection{October 23rd}
Today I learned a reason that monogenic extensions are, in a sense, kind of rare. Fix $K$ a number field. In a sense, for some $\alpha$ with minimal polynomial $f(x),$ the index $[\mathcal O_K:\ZZ[\alpha]]$ provides a measuring tool for how far away from a monogenic extension we are, and the same index describes the discrepancy between the Dedekind-Kummer factorization algorithm for ideals and actual prime-splitting. Namely, the condition that $p\nmid[\mathcal O_K:\ZZ[\alpha]]=1$ would be always satisfied in a monogenic extension, so we're always safe. But the Dedekind-Kummer factorization algorithm is often too nice to properly work.

For example, if $(2)$ splits completely in $\mathcal O_K$ where $K$ is not a quadratic field, then we immediately get a contradiction. Indeed, by Dedekind-Kummer says that $(2)$ splitting completely is the same as $f(x)\pmod2$ splitting completely, but there are only two possible roots for $f(x)$ to have, so splitting completely just isn't possible.

I guess I should provide a more concrete example. As a lemma to quadratic reciprocity, we show that for prime $n\mid p-1,$ $q$ is an $n$th power$\pmod p$ if and only if $q$ splits splits completely in the field of degree $n$ between $\QQ$ and $\QQ(\zeta_q).$ So, $(2)$ will split completely in the cubic subfield of $\QQ(\zeta_p)$ if and only if $2$ is a cube$\pmod p,$ where $p\equiv1\pmod3.$ Say, $4^3\equiv2\pmod{31}$ implies that the cubic subfield of $\QQ(\zeta_{31})$ named $F_3$ provides an example of a non-monogenic extension.

I'm not currently sure how to extract this subfield. Alec suggested considering the trace of some element because we know the Galois group. Indeed, note that
\[\sum_{k=0}^{30}\zeta_{31}^{k^3}\]
will be preserved by the Galois subgroup of $F_3,$ which is corresponds to the set of cubes in $(\ZZ/31\ZZ)^\times.$ Sage says that the minimal polynomial of this thing is $x^3 - 93x - 124,$ which is not at all clear to me, but sure. It follows that
\[F_3=\QQ(\alpha),\qquad\alpha^3-93\alpha-124=0.\]
That's a concrete field, but it still doesn't feel super satisfying because I didn't really generate it---I just handed it to Sage, and Sage will do the work.

\subsubsection{October 24th}
Today I learned the proof of Wedderburn's Theorem showing that finite domains $D$ are all fields. Namely, taking not-necessarily-commutative ring axioms plus the integral domain axiom, we must have a field if finite. Quickly, every nonzero element $a\in D\setminus\{0\}$ has an inverse by considering $\langle a\rangle,$ which by pigeonhole must eventually overlap with itself. As soon as $a^x=a^y$ with $x>y,$ then integral domain lets us cancel to say
\[a\cdot a^{x-y-1}=1,\]
providing us with our inverse.

The hard part, then, is showing that $D$ is commutative. The proof is actually quite nice. There's an algebraic step, which is basically writing down the class equation, and then there's an analytic step, which is basically bounding cyclotomic polynomials.

Because we're interested in commutativity information, we'll use the conjugacy class equation, under multiplication. Namely, make $D^\times$ act on itself by conjugation. Orbits of individual elements are conjugacy classes (by definition), so we get to write
\[|D^\times|=\sum_{\mathcal C\text{ class}}|\mathcal C|.\]
The elements with conjugacy class size $1$ are the center $Z(D^\times),$ so
\[|D^\times|=|Z(D^\times)|+\sum_{\mathcal C\text{ nontrivial class}}|\mathcal C|.\]
Considering the orbit-stabilizer theorem, we write this as
\[|D^\times|=|Z(D^\times)|+\sum_{[x]\text{ nontrivial class}}|D^\times x|.\]
But the number of elements in the orbit of $x$ is the index of $D^\times$ of the subgroup of $D^\times$ fixing $x$ (by the Orbit-Stabilizer Theorem), and the set of elements fixing $x$ is its centralizer. So we get to write
\[|D^\times|=|Z(D^\times)|+\sum_{[x]\text{ nontrivial class}}\frac{|D^\times|}{|C(x)|}.\]

We can actually put some values on each of the quantities of this class equation. Namely, using strong induction on the size of our domain (and that $2$-element unitary rings are fields as our base), we can assume that all finite domains smaller than $D$ are fields. Now, $Z(D)$ already commutes with itself, so it's a finite field of size $q.$ It follows that we can view $D$ and $C(x)$ for any $x\in D^\times\setminus Z(D^\times)$ both as $Z(D)$-vector spaces, of size $q^n$ and $q^d$ for positive integers $n$ and $d.$ Plugging this all in to the conjugacy class equation (and remembering to convert to the multiplicative groups) gives
\[q^n-1=(q-1)+\sum_{[x]\text{ nontrivial class}}\frac{q^n-1}{q^d-1}.\]
Note that $C(x)\subsetneq D^\times$ implies that $d<n.$ In particular, $C(x)=D^\times$ is equivalent to $x\in Z(D).$

That is the conjugacy class equation step. However, note that the cyclotomic polynomial $\Phi_n(x)$ divides $x^n-1$ and even each $\frac{x^n-1}{x^d-1}$ because $d<n.$ Throwing this at our class equation tells us
\[\Phi_n(q)~\left|~\left(q^n-1\right)-\sum_{[x]\text{ nontrivial class}}\frac{q^n-1}{q^d-1}=q-1\right.\]
as polynomials in $q$ and hence also for integers $q.$ But if $n>1,$ then
\[|\Phi_n(q)|=\prod_{\gcd(k,n)=1}\left|q-\zeta_n^k\right|>|q-1|\]
because of the placement of $\zeta_n^\bullet$ on the unit circle. This conflicts with $\Phi_n(q)\mid q-1,$ so we must have $n=1$ instead. Thus, $D=Z(D),$ and we're done here.

\subsubsection{October 25th}
Today I learned the definition of the categorical kernel. We begin with the definition of zero morphisms. It's zero morphisms plural because we define zero morphisms to be a set of morphisms in $\mathcal C$ that satisfy the property
\[0\circ a=b\circ 0=0\]
for any other morphisms $a$ and $b.$ Explicitly, every pair of objects $A,B\in\mathcal C$ is equipped with a zero morphism $0_{AB},$ and for any additional morphism $\varphi:B\to C,$ we have
\[\varphi\circ 0_{AB}=0_{AC}.\]
Similarly, for any additional morphism $\varphi:C\to A,$ we have
\[0_{AB}\circ\varphi=0_{CB}.\]
In other words, the following pair of diagrams always commute.
\begin{center}
    \begin{tikzcd}
        A \arrow[r, "0_{AB}"] \arrow[rd, "0_{AC}"'] & B \arrow[d] &  & A \arrow[r] \arrow[rd, "0_{AC}"'] & B \arrow[d, "0_{BC}"] \\
                                                    & C                      &  &                                              & C                    
    \end{tikzcd}
\end{center}
As an example, the category of groups has these very convenient morphisms $0:G\to G'$ by $g\mapsto e,$ which satisfy this condition. Indeed, composing any morphism before this still sends everything to the identity, and doing anything afterwards is doomed just to send the identity to the identity.

To define the kernel, now, we start with a function $f:A\to B.$ Then the kernel is an object $K$ with a morphism $k:K\to A$ to be thought of an inclusion mapping. We also require that $K\to A\to B=0_{KB}$ is the zero morphism and that if there's a parallel mapping $k':K'\to A$ with $K'\to A\to B=0_{K'B},$ then $k'$ factors uniquely through $k.$ In diagrams, this commutes.
\begin{center}
    \begin{tikzcd}
        A \arrow[rd, "f"]                     &   \\
        K \arrow[u, "k"] \arrow[r, "0_{KB}"'] & B
    \end{tikzcd}
\end{center}
And there is a unique $\varphi$ making this commute.
\begin{center}`
    \begin{tikzcd}
                                                                                                                                   & A \arrow[rd, "f"]                     &   \\
                                                                                                                                   & K \arrow[u, "k"] \arrow[r, "0_{KB}"'] & B \\
        K' \arrow[ru, "\varphi", dashed] \arrow[ruu, "k'", bend left, shift left] \arrow[rru, "0_{K'B}"', bend right, shift right] &                                       &  
    \end{tikzcd}
\end{center}
As alluded to, $K$ is usually what we refer to as ``the kernel'' because we would like it to be a subset of the input space; explicitly, we want $k$ to be an inclusion. For example, using our category of groups again, the typical kernel gets the job done. Indeed, $\ker(f)$ in groups is by definition the subgroup of $A$ which goes to the identity, and going to the identity is what our zero morphisms are.

And if we have another group so that $K'\to A\to B$ also goes to the identity, then this must be a subgroup of our original kernel, letting us back-construct the inclusion mapping into $K.$ This completes our checks.

\subsubsection{October 26th}
Today I learned an elementary proof that the set of primes which split completely in a Galois extension $K/\QQ$ of degree $n$ have density $1/n,$ only assuming some minor analytic facts about the Dedekind zeta function. Namely, this weaker statement does not require use of the Chebotarev Density Theorem. Basically, recall
\[\zeta_K(s)=\sum_{I\subseteq\mathcal O_K}\frac1{\op N(I)^s}=\prod_\mf p\frac1{1-\op N(\mf p)^{-s}}.\]
Noting that $\zeta_K(s)$ has a simple pole of rank $1$ at $s=1,$ we may say that $(s-1)\zeta_K(s)$ approaches some positive real number as $s\to1^+.$ We use this to get the result. In particular, we get to write
\[\lim_{s\to1^+}\frac{\log(\zeta_K(s))}{\log\left(\frac1{s-1}\right)}=1+\lim_{s\to1^+}\frac{\log((s-1)\zeta_K(s))}{\log\left(\frac1{s-1}\right)}=1.\]
Because this holds for $K=K$ and for $K=\QQ,$ we can divide the limits corresponding to each to get
\[\lim_{s\to1^+}\frac{\log(\zeta_K(s))}{\log(\zeta_\QQ(s))}=1.\]
But, akin to Dirichlet's Theorem, we can also expand
\[\log(\zeta_K(s))=\log\left(\prod_{\mf p}\frac1{1-\op N(\mf p)^{-s}}\right)=\sum_{\mf p}\sum_{k=1}^\infty\frac1{k\op N(\mf p)^{sk}}\]
because $-\log(1-x)=\sum_kx^k/k.$ We're going to get our result because the only terms that we care about in the above sum come from the rational primes that split completely.

This is an absolutely convergent series (all the terms are positive), so we are a bit liberal with our rearranging. In particular, we order this by rational primes $p$ so that this is
\[\log(\zeta_K(s))=\sum_p\sum_{\mf p/(p)}\sum_{k=1}^\infty\frac1{k\op N(\mf p)^{sk}}.\]
Dealing safely with the outer sum to start, we note that $e(\mf p/p)$ and $f(\mf p/p)$ are constant. The number of primes is therefore $\frac n{e(\mf p/p)f(\mf p/p)},$ giving $\op N(\mf p)=p^{f(\mf p/p)}.$ It follows that
\[\log(\zeta_K(s))=n\sum_p\frac1{e(\mf p/p)f(\mf p/p)}\sum_{k=1}^\infty\frac1{kp^{kf(\mf p/p)s}}.\]
Now we reduce this sum. We remark that only finitely many primes $p$ have $e(\mf p/p)>1,$ which will contribute a $-\log(1-p^{-f(\mf p/p)s})$ of inner sum, totalling only to $O(1)$ over all such (finitely many) primes. It follows
\[\log(\zeta_K(s))=O(1)+n\sum_{e(\mf p/p)=1}\frac1{f(\mf p/p)}\sum_{k=1}^\infty\frac1{kp^{kf(\mf p/p)s}}.\]
Further, note that we can also group all the $k>1$ terms into
\[\sum_{e(\mf p/p)=1}\frac1{f(\mf p/p)}\sum_{k=2}^\infty\frac1{kp^{kf(\mf p/p)s}}<\sum_{n=2}^\infty\sum_{k=2}^\infty\frac1{n^k}=\sum_{n=2}^\infty\frac1{n(n-1)}<\sum_{n=1}^\infty\frac1{n^2}<\infty.\]
So it follows
\[\log(\zeta_K(s))=O(1)+n\sum_{e(\mf p/p)=1}\frac1{f(\mf p/p)p^{f(\mf p/p)s}}.\]
But again, we can group all $p$ with $f(\mf p/p)>1$ by
\[\sum_{\substack{e(\mf p/p)=1\\f(\mf p/p)\ge2}}\frac1{f(\mf p/p)p^{f(\mf p/p)s}}<\sum_{n=2}^\infty\frac1{n^2}<\infty.\]
So we finally get to reduce our sum to
\[\log(\zeta_K(s))=O(1)+n\sum_{\substack{e(\mf p/p)=1\\f(\mf p/p)=1}}\frac1{p^s}.\]
However, this is a sum over $p$ completely splitting, which is equivalent to $e(\mf p/p)=f(\mf p)=1.$

To finish, we plug this into our limit from before, which tells us
\[1=\lim_{s\to1^+}\frac{\log(\zeta_K(s))}{\log(\zeta_\QQ(s))}=\lim_{s\to1^+}n\cdot\dfrac{\displaystyle\sum_{p\text{ splits completely}}\frac1{p^s}}{\displaystyle\sum_p\frac1{p^s}},\]
which rearranges to what we wanted, in terms of Dirichlet density.

\subsubsection{October 27th}
Today I learned a proof (after having skimmed one a few days ago) that a Galois extension is completely determined by its completely split primes. For a Galois extension $K/\QQ$ (or $L/K$; the proof extends), let $\op{Spl}(K/\QQ)$ be the set of primes which split completely. 

Now, the claim we show is that for two Galois extensions $K/\QQ$ and $L/\QQ,$ $K$ is a subfield of $L$ if and only if more primes completely split in $K$ than in $L,$ in the sense that $\op{Spl}(L/\QQ)\subseteq\op{Spl}(K/\QQ).$ Indeed, if $K$ is a subfield of $L,$ then a prime splitting completely in $L$ will split completely in $K$ for free because the ramification and inertial information are multiplicative in towers and so upper-bounded by $1.$

The converse is harder. The setup is to consider the field composite $LK,$ and the main idea is to use our connection between completely splitting primes and field invariants---namely, the Dirichlet density of completely split primes is the reciprocal of the degree. In symbols,
\[\frac1{[LK:\QQ]}=\lim_{s\to1^+}\dfrac{\displaystyle\sum_{p\in\op{Spl}(LK/\QQ)}\frac1{p^s}}{\displaystyle\sum_p\frac1{p^s}}\]
from our discussion yesterday. However, a prime splits completely in $LK$ if and only if it splits completely in both $L$ and $K$; we show the converse later, but the forward direction is discussed above. So the fact that the primes which split completely in $L$ also split completely in $K$ tells us that we only care about the primes which split completely in $L.$ Namely, $\op{Spl}(LK/\QQ)=\op{Spl}(L/\QQ),$ which implies
\[\frac1{[LK:\QQ]}=\lim_{s\to1^+}\dfrac{\displaystyle\sum_{p\in\op{Spl}(LK/\QQ)}\frac1{p^s}}{\displaystyle\sum_p\frac1{p^s}}=\lim_{s\to1^+}\dfrac{\displaystyle\sum_{p\in\op{Spl}(L/\QQ)}\frac1{p^s}}{\displaystyle\sum_p\frac1{p^s}}=\frac1{[L:\QQ]}.\]
To finish, we write
\[[LK:L]=\frac{[LK:\QQ]}{[L:\QQ]}=1,\]
so $L=LK,$ and $K$ is indeed a subfield of $L.$

To get the result we want, it follows that if $\op{Spl}(K/\QQ)=\op{Spl}(L/\QQ),$ then each is a subset of the other, so each of $K$ and $L$ are a subfield of the other, so $L=K.$ This is the statement that the set of completely split primes entirely determines the field extension, which is equivalent to the statement $K\mapsto\op{Spl}(K/\QQ)$ is an injective function, as shown.

I guess I should show that if a prime splits completely in $L$ and $K,$ then it also splits completely in the composite $LK.$ The trick is to take $M$ a normal extension containing $LK$ and then note that the decomposition field of a certain prime contains $L$ and $K$ (because the prime splits completely) and so contains $LK,$ implying that the prime splits completely. In particular, we are using the characterization that the decomposition field is the largest intermediate field such that the prime $\mf p/p$ has $e(\mf p/p)=f(\mf p/p)=1.$

\subsubsection{October 28th}
Today I learned an equivalent (Vakil's) definition of a kernel in a category with $0$ objects. Basically, if we have a $0$ object, then we can define our zero morphisms $0_{A0}$ and $0_{0A}$ as the only possible morphisms, for any object $A.$ But this kind of locks us into defining our zero morphism $0_{AB}$ by composing
\[A\stackrel{0_{A0}}\longrightarrow 0\stackrel{0_{0B}}\longrightarrow B.\]
Very quickly, for any morphism $C\to A,$ we do indeed have $0_{CB}$ after composing as $C\to A\to 0\to B,$ for $C\to A\to 0$ must be the unique mapping $C\to 0,$ so we had $C\to 0\to B$ all along. For the same reason, for any morphism $B\to C,$ the fact that we must have $0\to B\to C$ equal to $0\to C$ implies that the composite $A\to 0\to B\to C$ was $A\to 0\to C$ all along.

With this $0$ object in mind, we can define our kernel of a mapping $f:A\to B$ to be an object $K$ with a mapping $\iota:K\to A$ such that $\iota\circ f=0_{K0}$ and is universal with respect to this property. This is really the same diagram as we had last time, but having a $0$ object makes the definition must more natural. I'll repaste the diagram for completeness.
\begin{center}`
    \begin{tikzcd}
                                                                                                                                   & A \arrow[rd, "f"]                     &   \\
                                                                                                                                   & K \arrow[u, "\iota"] \arrow[r, "0_{KB}"'] & B \\
        K' \arrow[ru, "!", dashed] \arrow[ruu, "\iota'", bend left, shift left] \arrow[rru, "0_{K'B}"', bend right, shift right] &                                       &  
    \end{tikzcd}
\end{center}

\subsubsection{October 29th}
Today I learned the spigot algorithm for hexadecimal digits of $\pi.$ We begin with the formula
\[\pi=\sum_{k=0}^{\infty}\frac1{16^k}\left(\frac4{8k+1}-\frac2{8k+4}-\frac1{8k+5}-\frac1{8k+6}\right).\]
This is a computational result found by computer and whose proof I've read but is quite unexciting. For those who care, check \href{https://www.davidhbailey.com//dhbpapers/pi-quest.pdf}{here}.

Now, let's say we want the $n$th hexadecimal digit after the hexadecimal point. This is equivalent to computing $\floor{16^n\pi}\pmod{16}.$ For this, we actually look at 
\[\floor{16\left\{16^{n-1}\pi\right\}},\]
which is the same value, but we moved the modulo into the fractional part. This is nice because we can write
\[\left\{16^{n-1}\sum_{k=0}^\infty\frac1{16^k(8k+\ell)}\right\}=\left\{\sum_{k=0}^{n-1}\frac{16^{n-1}\pmod{8k+\ell}}{8k+\ell}+\sum_{k=n}^\infty\frac1{16^{n-1-k}(8k+\ell)}\right\}\]
because fractional parts allows us to ignore most of that $16^{n-1}.$ The spigot speed-up is to to note that we can actually opt to compute $16^{n-1}\pmod{8k+\ell}$ using modular exponentiation instead of computing $16^{n-1},$ implying that the entire process of computing this sum is merely $O(n\log n).$

The second sum is supposedly small, and we can compute digits until we're confident that it can't affect current digits. This can't really be theoretical guaranteed because there's always a change we hit some long string of $9$s and need to continually hit more digits from each of the sums, but this is quite unlikely assuming $\pi$ normal.

So we can quickly approximate
\[\left\{\sum_{k=0}^\infty\frac{16^{n-1}}{16^k(8k+\ell)}\right\}.\]
Noting that
\[\left\{16^{n-1}\pi\right\}=\left\{\left\{\sum_{k=0}^\infty\frac{4\cdot16^{n-1}}{16^k(8k+1)}\right\}-\left\{\sum_{k=0}^\infty\frac{2\cdot16^{n-1}}{16^k(8k+4)}\right\}-\left\{\sum_{k=0}^\infty\frac{16^{n-1}}{16^k(8k+5)}\right\}-\left\{\sum_{k=0}^\infty\frac{16^{n-1}}{16^k(8k+6)}\right\}\right\}\]
means that we can quickly approximate $\left\{16^{n-1}\pi\right\},$ which lets us quickly extract individual nibbles.

Honestly I'm pretty unimpressed by this algorithm, but it does qualify as ``spigot'' in that we got to ignore the nibbles before $n$ in our computation. However, we still have a few sums to worry about, and while we can be pretty sure in practice that these sums will not give us unbounded run times, it still makes me uncomfortable. So it goes.

\subsubsection{October 30th}
Today I learned a cute but unsurprising connection in some infinite Galois theory. Basically, we can ask about the Galois group
\[\op{Gal}\left(\bigcup_{k=0}^\infty\QQ(\zeta_{p^k})/\QQ\right).\]
Note that this kind of makes sense because each finite extension $\QQ(\zeta_{p^k})/\QQ$ is Galois, so we're more or less inducting up to infinity.

So what does this Galois group look like? Well, the motivating observation is that, really, we have
\[\QQ(\zeta_{p^\infty}):=\bigcup_{k=0}^\infty\QQ(\zeta_{p^k})\cong\varprojlim\QQ(\zeta_{p^\bullet}).\]
And we understand $\QQ(\zeta_{p^\bullet})$ really well. So we might hope that
\[\op{Gal}(\QQ(\zeta_{p^\infty})/\QQ)\stackrel?=\varprojlim\op{Gal}(\QQ(\zeta_{p^\bullet})/\QQ)=\varprojlim(\ZZ/p^\bullet\ZZ)^\times=\ZZ_p^\times.\]
To formalize this intuition, we begin by fixing some $\sigma\in\op{Gal}(\QQ(\zeta_{p^\infty})/\QQ).$ For any prime power $p^\bullet,$ we can restrict $\sigma$ to an automorphism $\sigma_\bullet$ of $\QQ(\zeta_{p^\bullet})/\QQ.$ Do note that we can identify $\sigma_\bullet$ with
\[\sigma_\bullet(\zeta_{p^\bullet})=\zeta_{p^\bullet}^{a_\bullet}\]
where $a_n\in(\ZZ/p^n\ZZ)^\times.$ This almost lets us map into $\ZZ_p^\times,$ but we have to also note that
\[\zeta_{p^n}^{a_{n+1}}=\zeta_{p^{n+1}}^{pa_{n+1}}=\sigma_{n+1}(\zeta_{p^{n+1}}^p)=\sigma(\zeta_{p^n})=\sigma_n(\zeta_{p^n})=\zeta_{p^n}^{a_n}.\]
So we know $a_{n+1}\equiv a_n\pmod{p^n},$ which lets us say that the sequence $(a_\bullet)$ generated by $\sigma$ does indeed live in $\ZZ_p^\times.$ Of course, we can map this backwards again. For any sequence $(a_\bullet)\in\ZZ_p^\times,$ we simply force
\[\sigma(\zeta_{p^\bullet})=\zeta_{p^\bullet}^{a_\bullet}\]
like before, and this is an automorphism of $\QQ(\zeta_{p^\infty}).$ I guess more formally, we'd have to show that $\sigma(\alpha+\beta)=\sigma(\alpha)+\sigma(\beta)$ and so on, but these hold because $\QQ(\alpha,\beta)$ is a subfield of some sufficiently large $\QQ(\zeta_{p^\bullet}),$ and $\sigma$ projects to an automorphism $\sigma_\bullet$ there. The fact that all of these projections agree to a single automorphism $\sigma$ when brought upwards is exactly the $a_{n+1}\equiv a_n$ discussed above.

So we get a nice statement $\op{Gal}(\QQ(\zeta_{p^\infty})/\QQ)\cong\ZZ_p^\times.$

\subsubsection{October 31st}
Today I learned the definition of Pr\"ufer groups. I think the most natural construction is as
\[\ZZ\left[\frac1p\right]/\ZZ=\left\{\left\{\frac a{p^n}\right\}:a,n\in\ZZ\right\}=\bigcup_{n=0}^\infty\left\{\sum_{k=0}^n\frac{a_k}{p^k}:a_k\subseteq[0,p)\cap\ZZ\right\}.\]
This definition lets us also say that this is $\QQ_p/\ZZ_p$ immediately, which is perhaps a more natural definition. Take $\exp(2\pi i\bullet)$ of this thing, we can say that this group is also isomorphic to
\[\left\{\zeta_{p^n}^k:k,n\in\ZZ\right\}.\]
With respect to category theory, we note that we have the embeddings $\ZZ/p^n\ZZ\to\ZZ/p^{n+1}\ZZ,$ so we can define $\QQ_p/\ZZ_p$ as
\[\varinjlim\ZZ/p^\bullet\ZZ.\]
This is seen most obviously by viewing this as $\ZZ[1/p]/\ZZ.$

Wikipedia defines these things as the unique $p$-group for which each element has exactly $p$ $p$th roots. This is indeed true of the group because
\[\frac a{p^n}=p\cdot\frac{a+kp^n}{p^{n+1}}\]
for any $k,$ of which there are $p$ options. I don't know why this would uniquely define a group; it's not even obvious to me why this property implies that the group is abelian.